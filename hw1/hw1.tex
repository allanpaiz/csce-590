\documentclass[11pt]{article}

\usepackage[margin=1in]{geometry}
\usepackage{amsmath, amssymb, amsthm}
\usepackage{tikz}
\usetikzlibrary{calc}
\usepackage{microtype}
\usepackage[hidelinks]{hyperref}
\usepackage{graphicx}

\setlength{\parindent}{0pt}
\setlength{\parskip}{4pt}

% ---- Header ----
\newcommand{\hwcourse}{CSCE 590 - 002}
\newcommand{\hwtitle}{Homework 1}
\newcommand{\hwname}{Allan Paiz}
\newcommand{\hwdate}{2/3/26}

\newcommand{\makehwheader}{
{\small
\textbf{\hwcourse}\hfill \hwname\\
\textbf{\hwtitle}\hfill \hwdate
}
\vspace{6pt}
\hrule
\vspace{10pt}
}

\newcommand{\problem}[1]{\vspace{1.5em}\textbf{Problem #1.}\ }
% \newcommand{\problem}[1]{\vspace{6pt}\textbf{Problem #1.}\ \textit{Solution.}\ }

\renewcommand\qedsymbol{$\blacksquare$}


\begin{document}

\makehwheader

% ---- Solutions ----
\section*{Part A}

\problem{1} My attempt at the simplex table method for example 3 by hand.
\begin{figure}[h]
    \centering
    \includegraphics[width=0.85\textwidth,height=15cm,keepaspectratio]{../MAX.png}
    \caption{Simplex table method for example 3 by hand.}
    \label{fig:a1}
\end{figure}

\newpage
\problem{2} I used the SimplexSolver to solve example 3. 
I couldn't figure out how to get the outputs of $s_1$ not to look like Figure 3.
\begin{figure}[h]
    \centering
    \includegraphics[width=0.65\textwidth]{../pa2b.jpg}
    \caption{Commands when running script generates Figure 3.}
    \label{fig:a2}
\end{figure}
\vspace{1em}
\begin{figure}[h]
    \centering
    \includegraphics[width=0.65\textwidth]{../pa2.jpg}
    \caption{Final lines of script generated document.}
    \label{fig:a3}
\end{figure}


\section*{Part B}
\problem{1}
We plot the following minimization problem:
\[
\begin{array}{rrl}
	\min_{x_1,x_2} & f = -2x_1 - x_2 \\ 
	\text{s.t.} & 2x_1 + \frac{8}{3}x_2 &\leq 4 \\
				  & x_1 + x_2 &\leq 2 \\
				  & 2x_1 &\leq 3 \\
				  & x_1, x_2 &\geq 0
\end{array}
\]
\vspace{-2em}
\begin{figure}[h]
    \centering
    \includegraphics[width=0.65\textwidth]{../Figure_b1.png}
    \caption{Minimization problem from example 3, lecture 1.}
    \label{fig:b1}
\end{figure}

\problem{2} We plot the following minimization problem:
\[
\begin{array}{rrl}
\min_{x, y} & f(x, y)=(x-a)^2+(y-b)^2 & \\
\text { s.t. } & (x-8)^2+(y-9)^2 & \leq 49 \\
& x+y & \leq 24 \\
& 2 & \leq x \leq 13 \\
\end{array}
\]
\vspace{-1.5em}
\begin{figure}[h]
    \centering
    \includegraphics[width=0.65\textwidth]{../Figure_b2.png}
    \caption{Minimization problem from example 7, lecture 1.}
    \label{fig:b2}
\end{figure}

\problem{3} We plot the graph for $f(x) = 3x^4 - 4x^3 + 1$, visualizing the horizontal point of inflection. \\
Let $f(x) = 3x^4 - 4x^3 + 1$, we want to solve $f'(x) = 0$.
We find the first derivative of $f(x)$, $f'(x) = 12x^3 - 12x^2 = 12x^2(x-1)$, thus $x=0$ and $x=1$ are inflection points, we plot $x=0$.
\vspace{-.5em}
\begin{figure}[h]
    \centering
    \includegraphics[width=0.65\textwidth]{../Figure_b3.png}
    \caption{Minimization problem from example 7, lecture 1.}
    \label{fig:b3}
\end{figure}

\section*{Part C}

\problem{1} Evaluate, using definitions of superposition principle and homogeneity, if the following functions are linear or nonlinear. \\ [1em]
A function $f : \mathbb{R} \rightarrow \mathbb{R}$ is linear if $\forall x,y \in \mathbb{R}$ and scalars $a,b \in \mathbb{R}$, $f(ax + by) = af(x) + bf(y)$. \\
That is, it must satisfy both:
\vspace{-1em}
\begin{enumerate}
    \item \textbf{Superposition:} $f(x + y) = f(x) + f(y)$, $\forall x,y$.
    \item \textbf{Homogeneity:} $f(\alpha x) = \alpha f(x)$, $\forall \alpha,x$.
\end{enumerate}
\vspace{1em}
\text{} \quad \textbf{(a)} $f(x) = |x|$
\begin{proof} We will show $f$ is not linear by counterexample.
    Let $x = 1$ and $y=-1$, then $f(x+y) = |1+(-1)|=0$, but $f(x) + f(y) = |1| + |-1| = 2$.
    Thus, $f(x + y) \neq f(x) + f(y)$ so we conclude $f$ is nonlinear.
\end{proof}
\text{} \quad \textbf{(b)} $f(x) = x^2 + 2x +2$
\begin{proof} We will show $f$ is not linear by counterexample.
    Let $x = 1$ and $y=-1$, then $f(x+y) = (0)^2 + 2(0) + 2 = 2$,
    but $f(x) + f(y) = [(1)^2 + 2(1) + 2] + [(-1)^2 + 2(-1) + 2] = 5 + 1 = 6$.
    Thus, $f(x + y) \neq f(x) + f(y)$ so we conclude $f$ is nonlinear.
\end{proof}
\text{} \quad \textbf{(c)} $f(x) = \frac{1}{x}$
\begin{proof} We will show $f$ is not linear by counterexample.
    Let $x = 1$ and $y = 2$, then $f(x+y) = \frac{1}{3}$, but $f(x) + f(y) = 1 + \frac{1}{2} = \frac{3}{2}$.
    Thus, $f(x+y) \neq f(x) + f(y)$ so we conclude $f$ is nonlinear.
\end{proof}
\text{} \quad \textbf{(d)} $f(x,y) = \sin(xe^y)$
\begin{proof} We will show $f$ is not linear by counterexample.
	Let $x = 1$ and $y = 0$, and $\alpha = 2$ then $f(\alpha x, \alpha y) = f(2,0) = \sin(2e^0) = \sin(2)$ and $\alpha f(x,y) = 2f(1,0) = \sin(1e^0) = 2\sin(1)$.
	Notice, homogeneity fails since $\sin(2) \neq 2\sin(1)$, thus $f(\alpha x, \alpha y) \neq \alpha f(x,y)$ so we conclude $f$ is nonlinear.
\end{proof}
\text{} \quad \textbf{(e)} $f(x,y) = \cos(x)y + x^2y$
\begin{proof} We will show $f$ is not linear by counterexample.
	Let $x = 1$ and $y = 2$, and $\alpha = 2$ then $f(\alpha x, \alpha y) = f(2,4) = 4\cos(2) + 16$ and $\alpha f(x,y) = 2f(1,2) = 2(2\cos(1) + 2) = 4\cos(1) + 4$.
	Notice, homogeneity fails since $4\cos(2) + 16 \neq 4\cos(1) + 4$, thus $f(\alpha x, \alpha y) \neq \alpha f(x,y)$ so we conclude $f$ is nonlinear.
\end{proof}

\newpage
\problem{2} Find the local and global minimizers and maximizers, if they exist, for the following functions: \\ [1em]
\text{} \quad \textbf{(a)} $f(x) = x + \sin(x)$ \\ [1em]
We begin by finding the first derivative of $f(x)$, $f'(x) = 1 + \cos(x)$, we want to find the critical points which satisfy $f'(x) = 0$.
\[
\begin{aligned}
    f'(x) = 1 + \cos(x) \\
          1 + \cos(x) &= 0 \\
          \cos(x) &= -1 \\
          x &= \pi.
\end{aligned}
\]
Thus, $x = \pi$ is a critical point, more precisely $\forall x = k \pi,$ with $ k = (2m+1), m \in \mathbb{Z}$ are critical points.

Next we perform the second derivative test on $f(x)$, $f''(x) = - \sin(x)$.
Notice $\forall x = k \pi$, $f''(x) = - \sin(k \pi) = 0$, thus the second derivative test is inconclusive.

We return to $f'(x) = 1 + \cos(x)$, we know $f'(x) = 0$ when $x = k \pi$.
Notice $1 + \cos(x) \in [0,2]$ since $\cos(x) \in [-1,1]$, thus $1 + \cos(x) > 0$ for all $x > k \pi$.
Since $f'(x)$ is increasing on both sides we conclude there is no local minimizer or maximizer.

Now notice,
\[ \lim_{x\to\infty} f(x) = \infty \quad \text{ and } \quad \lim_{x\to -\infty} f(x) = -\infty \]

Thus we conclude $f(x)$ has no global minimizer or maximizer.

\vspace{2em}
\text{} \quad \textbf{(b)} $f(x) = (2x_1 - x_2)^2 +(x_2 - x_3)^2 + (x_3 - 1)^2 $ \\ [1em]
We begin by finding the gradient of $f(x_1,x_2,x_3)$ and solving $\nabla f = 0$.
\[
\begin{aligned}
f(x_1,x_2,x_3) &= (2x_1-x_2)^2 + (x_2-x_3)^2 + (x_3-1)^2,\\[0.5em]
\frac{\partial f}{\partial x_1} &= 2(2x_1-x_2)\cdot 2 = 8x_1 - 4x_2,\\
\frac{\partial f}{\partial x_2} &= 2(2x_1-x_2)(-1) + 2(x_2-x_3) = -4x_1 + 4x_2 - 2x_3,\\
\frac{\partial f}{\partial x_3} &= 2(x_2-x_3)(-1) + 2(x_3-1) = -2x_2 + 4x_3 - 2. \\
\nabla f(x_1,x_2,x_3) &=
\begin{bmatrix}
8x_1-4x_2\\
-4x_1+4x_2-2x_3\\
-2x_2+4x_3-2
\end{bmatrix}.
\end{aligned}
\]

We now solve $\nabla f = 0$,
\[
\begin{bmatrix}
8x_1-4x_2\\
-4x_1+4x_2-2x_3\\
-2x_2+4x_3-2
\end{bmatrix} x = [0, 0, 0] ^T \]

Therefore $(x_1,x_2,x_3)=\left(\frac12,1,1\right)$ is a critical point.

Next we perform the second derivative test using the Hessian.
\[
\nabla^2 f(x_1,x_2,x_3)=
\begin{bmatrix}
8 & -4 & 0\\
-4 & 4 & -2\\
0 & -2 & 4
\end{bmatrix}.
\]
We check the leading principal minors,
\[
\Delta_1 = 8 > 0,
\]
\[
\Delta_2 =
\det\begin{bmatrix}8&-4\\-4&4\end{bmatrix}=32-16=16>0,\qquad
\Delta_3 =
\det\begin{bmatrix}8&-4&0\\-4&4&-2\\0&-2&4\end{bmatrix}= 96-64 =32>0.
\]

Thus $\nabla^2 f$ is positive definite, so $f$ is strictly convex and the critical point
$\left(\frac12,1,1\right)$ is a strict local minimizer by Theorem 22 in the lectures.
Hence $\left(\frac12,1,1\right)$ is also the unique global minimizer.

Then, $f$ has no local or global maximizer, since $\nabla^2 f$ is positive definite, $f$ is convex so it has no local maxima by Theorem 16 in the lectures, and
\[
\lim_{x_1,x_2,x_3\to\infty} f(x_1,x_2,x_3)=\infty,
\]
so $f$ is unbounded above and has no global maximizer.

\vspace{1em}
\problem{3} Show that no matter what value of $a$ is chosen, the function
$f(x_1,x_2) = x_{1}^{3} - 33ax_1 x_2 + x_{2}^{3}$ has no global maximizers.
Determine the nature of the critical points of this function for all values of $a$.

\vspace{0.5em}
First, we show $f$ has no global maximizer $\forall a\in\mathbb{R}$.
Let $x_2 = 0$ . Then $f(x_1,0)=x_1^3$, then as $x_1\to\infty$, $f(x_1,0) = \infty$.
Thus $f$ has no global maximizer for any value of $a$.

\vspace{0.75em}
Next, we find the critical points by solving $\nabla f = 0$.
\[
\begin{aligned}
f(x_1,x_2) &= x_1^3 - 33a x_1 x_2 + x_2^3,\\[0.25em]
\frac{\partial f}{\partial x_1} &= 3x_1^2 - 33a x_2,\\
\frac{\partial f}{\partial x_2} &= -33a x_1 + 3x_2^2. \\
\nabla f(x_1,x_2) &=
\begin{bmatrix}
3x_1^2 - 33a x_2\\
3x_2^2 - 33a x_1
\end{bmatrix}.
\end{aligned}
\]
We solve $\nabla f=0$,
\[
\begin{aligned}
3x_1^2 - 33a x_2 &= 0,\\
3x_2^2 - 33a x_1 &= 0.
\end{aligned}
\]

If $a=0$, Then $f(x_1,x_2)=x_1^3+x_2^3$ and
\[
 \nabla f(x_1,x_2) = \begin{bmatrix}3x_1^2\\3x_2^2\end{bmatrix} x = \begin{bmatrix}0\\0\end{bmatrix}
\;\text{, thus}\;\; x_1=0,\;x_2=0.
\]
So $(0,0)$ is the only critical point when $a=0$.
Notice if $x_2=0$ we have $f(x_1,0)=x_1^3$, so $f(x_1, 0)$ can be positive or negative, thus $(0,0)$ is a saddle point when $a=0$.

\vspace{1em}
If $a\neq 0$. From $3x_1^2-33a x_2=0$ we get $x_2=\frac{x_1^2}{11a}.$ Substitute into $3x_2^2-33a x_1 = 3\left(\frac{x_1^2}{11a}\right)^2 - 33a x_1  = x_1\left(\frac{3x_1^3}{121a^2} - 33a\right)=0$.

Thus, if $x_1=0$, then $x_2=0$, or $\left( \frac{3x_1^3}{121a^2} - 33a \right) = 0$ provides $x_1 = 11a$ and $x_2=11a$.

So for $a\neq 0$ the critical points are $(0,0)$ and $(11a,11a)$.

\vspace{0.75em}
To classify the critical points we use the Hessian.
\[
\nabla^2 f(x_1,x_2)=
\begin{bmatrix}
6x_1 & -33a\\
-33a & 6x_2
\end{bmatrix}.
\]

At $(0,0)$,
\[
\nabla^2 f(0,0)=\begin{bmatrix}0 & -33a\\-33a & 0\end{bmatrix},\qquad
\det(\nabla^2 f(0,0)) = 0 - (-33a)^2 = -1089a^2 < 0.
\]
Thus $(0,0)$ is a saddle point for $a\neq 0$ which follows as shown above.

Then, at $(11a,11a)$,
\[
\nabla^2 f(11a,11a)=\begin{bmatrix}66a & -33a\\-33a & 66a\end{bmatrix},\qquad
\det(\nabla^2 f(11a,11a)) = (66a)^2 - (-33a)^2 = 3267a^2 > 0.
\]
Since the determinant is positive, if $a>0$, then $66a>0$ and $(11a,11a)$ is a local minimizer. If $a<0$, then $66a<0$ and $(11a,11a)$ is a local maximizer.

\problem{4} Find the critical points for $f(x_1, x_2) = x_{1}^{5} - x_1 x_{2}^{6}$.
Is the critical point at origin, $(0,0)$, a local minimizer or maximizer?

\vspace{0.5em}
We begin by finding the gradient and solving $\nabla f = 0$.
\[
\begin{aligned}
f(x_1,x_2) &= x_1^5 - x_1x_2^6,\\[0.25em]
\frac{\partial f}{\partial x_1} &= 5x_1^4 - x_2^6,\\
\frac{\partial f}{\partial x_2} &= -6x_1x_2^5.
\end{aligned}
\]
Thus,
\[
\nabla f(x_1,x_2)=
\begin{bmatrix}
5x_1^4 - x_2^6\\
-6x_1x_2^5
\end{bmatrix}.
\]
We solve $\nabla f=0$:
\[
\begin{aligned}
5x_1^4 - x_2^6 &= 0,\\
-6x_1x_2^5 &= 0.
\end{aligned}
\]
From $-6x_1x_2^5=0$, we have $x_1=0$ or $x_2=0$, by the zero product property.

If $x_1=0$, then $-x_2^6 =0$, $x_2=0$. \\
If $x_2=0$, then $5x_1^4=0$, $x_1=0$. Thus, $(0,0)$ is a unique critical point.

\vspace{0.75em}
Notice $f(0,0)=0$, then consider $f(x_1,0)=x_1^5.$
For $x_1>0$, $f(x_1,0)>0$, and for $x_1<0$, $f(x_1,0)<0$.
Thus, $(0,0)$ is neither a local minimizer or maximizer, it is a saddle point.


\problem{5}

\text{} \quad \textbf{(a)} Describe an example or scenario where you can incorporate optimization principles.

An example can be minimizing the amount of interest paid on multiple balances with different interest rates.
For example 3 credit cards, you must meet the minimum payments, on a fixed income/budget.
To make it more in-depth you could also minimize the time to pay off the balances, also under a fixed income/budget.

\vspace{0.4em}
\text{} \quad \textbf{(b)}  State the described optimization problem similar to the examples (Ex 1,2,5,8) in Lecture 1 material by adding the information needed to formulate the optimization problem mathematically.

Suppose there are 3 credit cards, each with different simple interest rates, current balance, and required minimum monthly payment.
Each month $t$ the income is $P = \$1000$ which must be divided to pay off the 3 credit cards.
Over a 12 month span the goal is to minimize the total amount of interest paid, under the constraints of monthly income and the minimum monthly payment due.
For simplicity we are not concerned with paying off the balances, we will use the following:

\begin{center}
\begin{tabular}{|c|c|c|c|}
\hline
Card ($i$) & Interest rate ($r$) & Balance $(b)$ & Minimum  ($m$) \\
\hline
A & 0.0208 & \$21,00 & 75 \\
B & 0.0150 & \$15,000 & 65 \\
C & 0.0104 & \$32,000 & 112 \\
\hline
\end{tabular}
\end{center}

Let $x_{i,t}$ denote the payment made to card $i$ in month $t$. 
Let $b_{i,t}$ denote the balance each month by adding the interest and then subtracting the payment made. 
The objective is to choose payments $\{x_{i,t}\}$ over 12 months to minimize total interest paid over the 12-month period, subject to monthly income and meeting each card's minimum payment.
\vspace{0.6em}

\vspace{0.6em}
\text{} \quad \textbf{(c)} Formulate/Derive the mathematical model of the stated optimization problem.
\[
\begin{aligned}
\min_{x_{i,t},\,b_{i,t}} \quad 
& \sum_{t=1}^{12}\sum_{i=1}^{N} r_i\,b_{i,t} \\
\text{s.t.}\quad
& b_{i,t+1} = (1+r_i)b_{i,t} - x_{i,t}, && i=1,\dots,N,\;\; t=1,\dots,12,\\
& \sum_{i=1}^{N} x_{i,t} \le P, && \\
& x_{i,t} \ge m_i, && \\
& x_{i,t}, b_{i,t} \ge 0. \\
\end{aligned}
\]

\vspace {2em}
\section*{Appendix}
All files and code can be found \href{https://github.com/allanpaiz/csce-590/tree/main/hw1}{on Github}.

\url{https://github.com/allanpaiz/csce-590/tree/main/hw1}

\vspace {2em}
\section*{Sources}
\textbf{Use of AI} \quad During the completion of this assignment, I used ChatGPT 2.5 (OpenAI) as an assistive tool.
I reviewed and edited all AI assisted outputs, and take full responsibility for the accuracy of my work. 

\end{document}
